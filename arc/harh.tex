\documentclass{article}

\setlength\parskip{0.4em}

\usepackage[utf8]{inputenc}
\usepackage[spanish]{babel}
\usepackage[T1]{fontenc}
\usepackage{mathpazo}
\usepackage{csquotes}
\usepackage{siunitx}
\usepackage{graphicx}
\usepackage{hyperref}
\hypersetup{
	colorlinks=true,
	linkcolor=blue,
	filecolor=magenta,      
	urlcolor=cyan,
}
\usepackage{caption}
\captionsetup[figure]{labelformat=empty}

\title{Hedonismo, ascetismo y la respuesta hermética}

\author
{
	Luke Smith\\\\
	\small
	{
		Traducido del
		original\footnote{\url{https://lukesmith.xyz/articles/poetic}.}
		\ en inglés por J. D. Gambín\footnote{La versión web de esta
		traducción está en \url{https://jdgambin.github.io/tr/harh.html}.}
	}
}

\date{\small{28 de diciembre de 2020}}

\begin{document}
	\maketitle

	El mundo moderno más o menos te da la elección filosófica
	exclusiva entre \textit{hedonismo} o \textit{ascetismo}.
	En realidad, nunca lo escuchas en estos términos, pero así es.

	\textbf{El hedonismo es vivir por placer}. Tu estilo de vida automático
	se basa en comer lo que te plazca, ver Netflix y jugar videojuegos sin
	importar qué tan tarde es. Miras porno, te masturbas, tienes sexo tanto
	como puedes, y las consecuencias de esto son hechos de la vida fuera de
	tu control o por los que vale la pena sufrir. Puede que no uses drogas
	porque te preocupa el daño hedónico que pueden causar, pero al menos
	eres «relajado» con la gente que lo hace. En un nivel básico, la
	sociedad moderna es hedonista porque, más o menos abiertamente, sostiene
	como mayor valor moral lo que puede estimular más a las personas. Sabes
	que este es el caso porque cualquiera que \textit{condene} el
	comportamiento hedónico será inmediatamente juzgado como
	\textquote{moralista}.

	\textbf{El ascetismo es supuestamente la alternativa
	\textquote{inteligente}}. Ascetismo es rechazar el placer, la vida normal
	y cualquier otra cosa que se disfrute en el mundo como moralmente
	inferior a algún ideal no físico más elevado. El budismo, el cual
	rechaza el mundo físico, se ha convertido en una filosofía meme popular
	en Occidente y es altamente ascético. Los veganos son ascéticos:
	abandonan la vida básica por sus propios principios y los veganos
	intensos eventualmente comienzan a hablar sobre \textquote{trascender},
	\textquote{vibraciones} y disparates. Mira los movimientos contra el
	calentamiento global, se ajustan perfectamente y casi neuróticamente a
	esta categoría. El ascetismo viene en muchas formas hoy día, pero
	siempre es una reacción a las indulgencias del hedonismo.

	\newpage

	\section*{La Perspectiva Poética}

	\begin{figure}[ht]
	\centering
	\includegraphics[scale=.6]{hermes.png}
	\caption*{Hermes Trismegistus
	\protect\footnotemark{}
	autor del \textit{Corpus Hermeticum}.}
	\end{figure}
	\footnotetext{\url{https://renaissanceastrology.com/hermestrismegistus.html}.}

	La Perspectiva Poética es la solución. No te preocupes, no tiene nada
	que ver con la poesía.

	La palabra griega \textit{poesía/poético} proviene de \textit{hacer,
	crear, producir}. La palabra \textquote{poesía} originalmente
	significaba algo como \textquote{producción creativa}.

	Este punto de vista está ligado al Platonismo temprano y al monoteísmo.
	El universo físico es una creación o manifestación de
	\textquote{el Único} o \textquote{la Fuente}, o realmente,
	\textit{Dios}. Dios es el creador supremo, y un individuo es
	\textbf{bueno} en la medida en que \textit{él refleja esta tendencia
	creativa de Dios}. Lo vemos expuesto en el \textit{Corpus Hermeticum}:

	\blockquote[Asclepio\footnote{\url{http://www.gnosis.org/library/grs-mead/TGH-v2/th205.html}.}
	(17), Corpus Hermeticum\footnote{\url{http://www.gnosis.org/library/hermet.htm}.} \cite{uno}]
	{Y el otro nombre de Dios es «Padre». Es llamado Padre por ser el
	hacedor o progenitor de todas las cosas; pues es a un padre a quien le
	corresponde generar.}

	En la Perspectiva Poética, \textbf{el fin moral superior es la
	creación}. Esto puede ser:

	\begin{itemize}
	\item afectar el mundo,
	\item mejorar lo que hay a tu alrededor,
	\item tener hijos,
	\item ganar dinero no para gastarlo en placeres, sino para hacer algo
		nuevo y grandioso con ello,
	\end{itemize}

	\newpage

	\begin{itemize}
	\item escribir o hacer algo útil o edificante para otros,
	\item aclarar conceptos erróneos que se interponen en el camino de otras
	personas para lograr estas cosas.
	\end{itemize}

	En la visión Poética del mundo, el hedonismo es malvado porque gasta
	energía potencialmente creativa en actividades estériles. Acumular logros
	de videojuegos que nadie conocerá o dará importancia a excepción de ti,
	mirar pornografía, perseguir relaciones efímeras, desperdiciar
	impulsivamente el tiempo navegando la internet y jugando mezquinamente
	con las redes sociales.

	Esta búsqueda pasiva e impulsiva del placer reduce la capacidad de
	las personas para vivir como está intencionado, en cambio, las hace
	prisioneras de sus lujurias y conveniencias:

	\blockquote[Sobre la Mente
	común\footnote{\url{http://www.gnosis.org/library/grs-mead/TGH-v2/th225.html}.}
	(4), Corpus Hermeticum\footnote{\url{http://www.gnosis.org/library/hermet.htm}.}
	\cite{dos}]
	{«Por otra parte, todas las almas humanas que no lograron tener a
	la Mente como piloto, sufren la vida de los animales
	irracionales, pues la Mente las ayuda a que se consoliden las
	pasiones a las obedecen sin razón a sus cóleras y sin razón no
	se cansan de desear ni se hastían de los vicios. Por eso el
	instinto colérico y la pasión del deseo son los vicios máximos.
	Estas son las almas a las que Dios impuso la Mente como juez y
	verdugo.»}

	El hedonismo es adicionalmente dañino porque \textit{ni siquiera es
	hedónico}. Es torpe y autodestructivo. Vive desde los 16 a los 23
	años jugando videojuegos, masturbándote y fumando hierba, y habrás
	destruido tu capacidad para disfrutar la vida, el sexo y tener
	interacciones normales con gente normal. Tu capacidad para el disfrute
	termina y caes en el ascetismo como mecanismo psicológico de defensa.

	\textbf{El ascetismo es igual de malvado} porque ve este problema del
	estilo de vida hedónico y levanta las manos en señal de renuncia.
	Internaliza la mentira de que vivir de forma derrochadora y pecaminosa
	es obviamente lo más divertido; cuando ven que en realidad no se están
	divirtiendo, tiran el mundo entero por la borda.

	\textbf{De cualquier forma, la mayoría de los ascéticos son mentirosos}.
	Pretenden rechazar los placeres y las cosas mundanas, pero a menudo
	simplemente realizan su búsqueda de formas perversas o no convencionales.
	Hay hombres que se hacen llamar \textit{MGTOW} (\textit{hombres que
	siguen su propio camino}, por sus siglas en inglés) que
	\textquote{renuncian} a las mujeres. En realidad, la mayor parte de
	ellos son hombres desesperados adictos al porno que simplemente no
	pueden conseguir a la chica que quieren.

	\newpage

	Contrariamente a todo esto, una visión Poética propone que la
	vida más moral, y también la más placentera, es una donde constantemente
	estamos creando algo nuevo a partir de lo que se nos ha dado.
	En el pensamiento Hermético (y en el Cristiano) el hombre debe tener
	a Dios como el ideal a emular. Puesto que la principal hazaña de Dios
	es la creación a partir de la nada, nuestro objetivo es celebrar esta
	creación al hacer algo nuevo y productivo con los materiales básicos
	a nuestra disposición.

	El ascetismo ve al mundo material \textit{como un error o una ilusión},
	lo que lleva a la gente a rechazar la vida misma. El punto de vista
	Poético es que el mundo físico es una \textbf{reflexión de su estado
	espiritual}, y lo que haces en el mundo físico, refleja tu estatura
	espiritual.

	La visión Poética es \textit{algo} similar a la \textit{Voluntad de
	Poder} de Nietzsche, que fue un intento de unificar las ciencias humanas
	y materiales bajo la idea de que el ideal es maximizar nuestro efecto en
	el mundo externo. Aunque la Voluntad de Poder es un poco más ambivalente
	en el sentido moral; esta puede incluir la destrucción, mientras que el
	Poeticismo meramente valora el poder creativo.

	\section*{Las distracciones son literalmente malvadas}

	Es por esto que \textit{condeno altamente las actividades derrochadoras como
	los videojuegos, la pornografía y las redes sociales}. Son
	principalmente hábitos que desvían tus energías naturales hacia cosas
	absolutamente esteriles. Mucha gente me pregunta \textquote{¿qué puedo
	hacer para ser más productivo?}, y tengo que decir que lo más importante
	es \textbf{eliminar las distracciones y hábitos inertes}.

	Debido a los entornos de trabajo burocráticos y a la educación
	burocrática, hay muchas personas modernas que \textit{simplemente no
	saben lo que significa ser productivo}. La mayor parte de sus vidas
	puede consistir en intentar llenar el día con trabajos superficiales
	o papeleo. Dado que el agradable ritual normal de la producción creativa
	es desconocido para ellos, esto provoca un tipo de desorientación y los
	sentimientos de
	inferioridad\footnote{\url{https://lukesmith.xyz/files/unabomber.pdf}.}
	\cite{tres} que vienen con ello.

	Pero en verdad, vives en \textbf{un periodo ideal}, en el que puedes
	tener un efecto altamente impactante (y por tanto \textit{poético}) en
	el mundo usando las tecnologías de internet y el más alto estándar de
	vida material. El único truco es burlar las distracciones del hedonismo
	que te transforman en un consumidor pasivo y evitar la apatía del
	ascetismo.

	\newpage

	\renewcommand\refname{\large{Notas del traductor}}

	\begin{thebibliography}{9}
		\bibitem{uno} Cita extraída de
		\href{https://books.google.com.co/books/about/Corpus_Hermeticum.html?hl=es&id=It8Jo57qFVgC}
		{Corpus Hermeticum: y otros textos apócrifos} de Ed. Edaf
		(p. 46) traducido por Manuel Algora.
		
		\bibitem{dos} Cita (editada) extraída de
		\href{http://www.ricardoego.com/libros/Corpus%20Hermeticum%20-%20Trismegisto%20Hermes.pdf}
		{Corpus Hermeticum} (p. 20) traducido por J. Sanguinetti. 

		\bibitem{tres} Leer \textquote{La sociedad industrial y su
		futuro}
		(\href{https://isumatag.blogspot.com/p/la-sociedad-industrial-y-su-futuro.html}
		{leer reseña},
		\href{http://libgen.rs/book/index.php?md5=F50ABA1457C931AA22DD7BC84D5A90C9}
		{descargar libro}) de Theodore J. Kaczynski, traducido al español por
		Ediciones Isumatag.
	\end{thebibliography}
\end{document}
